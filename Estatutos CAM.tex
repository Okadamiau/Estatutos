\documentclass[letterpaper,11pt]{article}
% Packages
\usepackage[spanish]{babel}
\usepackage[utf8]{inputenc}
\usepackage[usenames,dvipsnames]{color}
\usepackage[margin=1in]{geometry}
\usepackage[usenames,dvipsnames]{color}
\usepackage[T1]{fontenc}
\usepackage{lmodern}
\usepackage{textcomp}
\usepackage{hyperref}
\usepackage{amsmath}
\usepackage{amsthm}
\usepackage{fancyhdr}
\usepackage{titlesec}

% Formatting
\hypersetup{colorlinks= true, linkcolor=black}
\spanishdecimal{.}
\tolerance=1000
\hyphenpenalty=1000
\setlength{\parindent}{0in}
\setlength{\parskip}{0.1in}
\setlength\textheight{8.2in}
\setlength\topmargin{-1in}
\lhead{\sc Estatutos Centro de Alumnos de Matemáticas\\Pontificia Universidad Católica de Chile}
\chead{}
\rhead{}
\lfoot{}
\cfoot{\thepage}
\rfoot{}
\fancypagestyle{plain}{%
\fancyhf{}
\cfoot{\thepage}
\lfoot{}
\renewcommand{\headrulewidth}{0pt}}
\renewcommand{\headrulewidth}{1 pt}
\renewcommand{\footrulewidth}{0 pt}
\headheight=75pt

\theoremstyle{plain}
\newtheorem{art}{Art.} % Para que salga Art.+Nro
\newtheorem{art_trans}{Art.}

%\titleformat{\section}[display]{\scshape \Large}{Título \thesection:}{}{}[]

% Enumeraciones
\renewcommand{\theenumi}{\roman{enumi}}
\renewcommand{\labelenumi}{\theenumi.}
\renewcommand{\theenumii}{\arabic{enumii}}
\renewcommand{\labelenumii}{(\theenumii)}
\renewcommand{\theenumiii}{\roman{enumiii}}
\renewcommand{\labelenumiii}{\theenumiii.}
\renewcommand{\theenumiv}{(\alph{enumiv})}
\newcommand{\HRule}{\rule{\linewidth}{0.5mm}}
\newcommand{\aref}[1]{\hyperref[#1]{\ref*{#1}}}
\newcommand{\aaref}[2]{\hyperref[#2]{\ref*{#1}, letra \ref*{#2}}}
\makeatletter \renewcommand\p@enumii{\theenumi, } \makeatother
\makeatletter \renewcommand\p@enumiii{\theenumii, } \makeatother
\makeatletter \renewcommand\p@enumiv{\theenumiii, } \makeatother

% Formalidades respecto al título, autor y fecha de modificación
\title{Estatutos}
\author{Centro de Alumnos de Matemáticas}
\date{Noviembre 2019}

% Inicio del documento
\begin{document}
\pagenumbering{gobble}
\thispagestyle{plain}
\vspace*{-75pt}

\begin{center}
	\begin{Large}
		{\bf
			ESTATUTOS DEL CENTRO DE ALUMNOS DE MATEMÁTICAS

			PONTIFICIA UNIVERSIDAD CATÓLICA DE CHILE
		}
	\end{Large}

	\vspace*{30pt}

\end{center}
\tableofcontents
\newpage
\pagenumbering{arabic}

\section{Declaración de Principios}\label{principios}
``Nosotros, la comunidad de estudiantes de matemática y estadística ratificamos nuestro compromiso con la defensa de los derechos e intereses de todos los estudiantes de la Facultad de Matemáticas, siendo rasgos fundamentales del centro de alumnos de matemáticas su naturaleza participativa, pluralista y el respeto a las decisiones de las mayorías. Junto con esto, manifestamos nuestro respeto a los valores cristianos.''

\section{Norma Generales}\label{normasGenerales}
\begin{art}\label{}
	Se establecen como normas estatutarias del centro de alumnos de matemáticas (CAM) las contenidas en los presentes estatutos.% TODO Redo, hay que considerar que como se denotará el CAM y la directiva
\end{art}

\begin{art}\label{}
	El CAM es el organismo que representa ante la comunidad universitaria de la Pontificia Universidad Católica de Chile (PUC) a:
	\begin{enumerate}
		\item Estudiantes de pregrado regulares de la facultad de matemáticas
		\item Estudiantes de postgrado regulares que anuncien su deseo de ser representados.
		\item Llamaremos a ``estudiantes representados'' al conjunto descrito por los puntos anteriores.
	\end{enumerate}
\end{art}

\begin{art}\label{finalidadesCAM}
	El CAM tiene por finalidades:
	\begin{enumerate}
		\item Velar por los derechos e intereses de los estudiantes representados.
		\item Promover, impulsar y desarrollar las actividades conducentes a la correcta formación, integración y recreación de los estudiantes representados.
		\item Procurar que la enseñanza de la Facultad de Matemáticas (desde ahora "la Facultad") sea de excelencia, creando profesionales comprometidos con la sociedad.
		\item Administrar los fondos puestos a disposición del CAM por su antecesor, por la Facultad y la FEUC. Asimismo, administrar los recursos que se generen mediante actividades, donaciones, entre otros.
	\end{enumerate}
\end{art}

\begin{art}\label{atribucionesEstudiantes}
	Son atribuciones de los estudiantes:
	\begin{enumerate}
		\item Manifestar libremente su opinión en las Asambleas Generales y otros espacios destinados por el CAM para tal efecto.
		\item Publicar información y/u opiniones en espacios destinados por el CAM, siempre y cuando el CAM no ponga objeciones.
		\item Participar en las actividades que organice el CAM.
		\item Exigir información sobre el trabajo del CAM en forma oral o escrita.
		\item Proponer proyectos al CAM.
		\item Solicitar espacios, equipos y/o materiales que sean del CAM para realizar actividades que brinden beneficios a la comunidad universitaria.
	\end{enumerate}
\end{art}

\begin{art}\label{deberesCAM}
	Son deberes de los miembros del CAM:
	\begin{enumerate}
		\item Reportar los derechos de cada uno de los estudiantes.
		\item Hacer efectivo su derecho a voto en las elecciones.
		\item Asistir a las asambleas generales al ser convocadas.
		\item Respetar la opinión y los espacios de democracia que se generen.
	\end{enumerate}
\end{art}

\begin{art}\label{definicionesOrganismos}
	Los organismos directivos del CAM son los siguientes:
	\begin{enumerate}
		\item Directiva: Está conformada por cinco miembros por cinco miembros. En orden:
		\begin{enumerate}
			\item Presidente
			\item Vicepresidente Interno
			\item Vicepresidente Externo
			\item Secretario General
			\item Tesorero
		\end{enumerate}
		\item Consejo de Delegados: Se conformará por los delegados de cada generación de cada carrera, un delegado de Postgrado y un delegado correspondiente a todos los estudiantes no considerados en los grupos anteriores. Las generaciones corresponden al año de ingreso a la carrera hasta cinco años antes del actual en caso de Estadística y hasta cuatro años en el caso de Matemáticas.
		\item Consejerías Académicas: Hay dos consejerías académicas, cada una corresponde a una carrera de pregrado, y cada una está conformada por un Consejero y un Subconsejero.
		\item Consejo Estudiantil: Está conformado por el Consejo de Delegados, las Consejerías Académicas, el Consejero Territorial y dos miembros de la Directiva del CAM. Los Consejos que se convoquen serán abiertos y podrá participar con derecho a voz todo estudiante que así lo desee.
	\end{enumerate}
\end{art}

\begin{art}
	Todo cargo se pierde:
	\begin{enumerate}
		\item Renuncia voluntaria
		\item Destitución por medios definidos en el artículo \ref{destituciones}
		\item Por expulsión de la Universidad.
		\item Abandono de la carrera.
		\item Por suspensión de la carrera de forma voluntaria y temporal.
		\item Finalización de estudios en la Facultad.
		\item Al terminar el semestre con promedio inferior a 4,0
	\end{enumerate}
\end{art}

\begin{art}
	Los estudiantes que estén en el proceso de cambio de carrera serán representados por el CAM hasta que el cambio se haga efectivo, además al hacerse efectivo el cambio el estudiante deberá informar este cambio a los respectivos centros de estudiantes.
\end{art}

\section{De las funciones y Atribuciones}\label{funcionesAtribuciones}
\begin{art}\label{funcionesDirectiva}
	Son funciones, atribuciones y deberes de la Directiva:
	\begin{enumerate}
		\item Desarrollar el plan de trabajo presentado y dar cuenta trimestral en las Asambleas Generales.
		\item La Directiva por tanto tendrá que convocar a lo menos dos asambleas por semestre.
		\item Crear y promover el desarrollo de actividades de orden estudiantil y/o de formación general e integral en el estudiantado.
		\item Crear y promover cargos participativos (i.e. vocalías) para el estudiantado.% TODO Rewrite, add stuff about vocalías
		\item Pronunciarse ante los problemas estudiantiles y sociales, e informar su postura al respecto.
		\item Representar ante autoridades universitarias, organismos superiores y la Federación de Estudiantes (FEUC), a los estudiantes representados.
		\item Administrar los bienes que estén a disposición del CAM.
		\item Convocar a sesiones ordinarias y extraordinarias de la Asamblea General.
		\item Informar a los estudiantes de las actividades realizadas y a realizar.
		\item Ningún miembro de la directiva podrá abstenerse en una votación de Asamblea General.
		\item Asistir a los Consejos de Federación, de Facultad y Estudiantil.
		\item En cuanto a la toma de decisiones, la directiva tendrá la facultad de sesionar cuantas veces estime conveniente para cumplir con las funciones que contemple el actual estatuto.
		\item Las decisiones serán tomadas  luego del acuerdo de la totalidad de la Directiva, o como los miembros de esta estimen conveniente.
		\item Todas las demás disposiciones que el presente estatuto estipule.
	\end{enumerate}
\end{art}

\begin{art}\label{funcionesPresidente}
	Son funciones del Presidente:
	\begin{enumerate}
		\item Tomar la representación del CAM tanto dentro como fuera de la universidad, de acuerdo a las finalidades del CAM.
		\item Garantizar la existencia de un vocero de las inquietudes de los estudiantes.
		\item Asegurar una estructura eficiente en las Asambleas Generales.
		\item Presidir reuniones de Directiva.
		\item Realizar que sea dos cuentas públicas cada semestre, una al principio y otra al final del semestre, donde se deberá entregar una evaluación con los hitos más importantes del semestre, junto con un estado financiero que refleje el gasto de dinero del CAM.
	\end{enumerate}
\end{art}

\begin{art}\label{funcionesVicepresidenteInterno}
	Son funciones del Vicepresidente Interno:
	\begin{enumerate}
		\item Apoyar al Presidente en las funciones que este deba cumplir con la comunidad de Matemáticas.
		\item Subrogar al Presidente en su ausencia.
		\item Velar por el cumplimiento de las funciones de los distintos miembros de la Directiva y de los miembros del Consejo Estudiantil.
		\item Encargarse de las relaciones entre la Directiva y las autoridades, profesores y funcionarios de la Facultad. Asimismo, deberá coordinar cualquier trabajo con el Consejo Estudiantil.% TODO Cambiar Encargarse por una mejor palabra
		\item Difundir toda la información que la directiva deba y/o quiera difundir por todos los medios oficiales.
	\end{enumerate}
\end{art}

\begin{art}\label{funcionesVicepresidenteExterno}
	Son funciones del Vicepresidente Externo:
	\begin{enumerate}
		\item Apoyar al Presidente en las funciones que este deba cumplir con la comunidad universitaria.
		\item Subrogar al Vicepresidente Interno en caso de ausencia de este.
		\item Encargarse de las relaciones entre la Directiva y las autoridades universitarias, la Federación, Movimientos Políticos, y las instituciones y organismos no pertenecientes a la universidad.% TODO Lo mismo que con Vicepresidente Interno
	\end{enumerate}
\end{art}

\begin{art}\label{funcionesSecretario}
	Son funciones del Secretario General:
	\begin{enumerate}
		\item Publicar en un lugar visible las citaciones a Asambleas Generales, además de publicar las actas de estas sesiones.
		\item Mantener el inventario al día y tenerlo a disposición de la Directiva cuando esta lo estime conveniente.
		\item A comienzos de año dar a conocer los estatutos que rigen al CAM, publicándolos en todos los medios oficiales.
		\item Recibir proyectos de reglamentos y de reformas a los estatutos, además de velar por el cumplimiento de estos.
		\item Tomar acta de las Asambleas y reuniones de Directiva tanto ordinarias como extraordinarias.
		\item Publicar las actas de asambleas en todos los medios oficiales.
	\end{enumerate}
\end{art}

\begin{art}\label{funcionesTesorero}
	Son funciones del Tesorero:
	\begin{enumerate}
		\item Llevar la contabilidad del CAM, entendiéndolo como el mantener el libro de ingresos y egresos al día, mantener todas la boletas del CAM y todo lo que compruebe las cuentas realizadas. Teniéndolos a disposición de la Directiva y miembros del Consejo Estudiantil cuando este lo estime conveniente.
		\item Presentar un informe trimestral de haberes.
		\item Administrar la cuenta de ahorros y/o corriente del CAM, si existiere.
		\item Administrar los fondos de becas del CAM.
		\item Evaluar los gastos y ganancias que impliquen los proyectos que se generen de parte de la comunidad.
	\end{enumerate}
\end{art}

\begin{art}\label{funcionesDelegados}
	Son funciones y atribuciones de los Delegados:
	\begin{enumerate}
		\item Ayudar a la directiva en las actividades que se realicen para la comunidad, tanto en la realización como en su difusión.
		\item Difundir información que el CAM entregue al nivel que pertenezca.
		\item Asistir a los Consejos Estudiantiles que se convoquen.
	\end{enumerate}
\end{art}
\begin{art}\label{atribucionesDelegados}
	Son atribuciones de los delegados:
	\begin{enumerate}
		\item Proponer y llevar a cabo proyectos aprobados por el Consejo Estudiantil
	\end{enumerate}
\end{art}

\begin{art}\label{funcionesConsejeriaAcademica}
	Son funciones y atribuciones de la Consejería Académica:
	\begin{enumerate}
		\item Ser defensor académico, y todo lo que ello conlleva, de los estudiantes representados. % TODO Cambiar con respecto a la situación de permanencia
		\item Asistir a los comités curriculares de la Facultad, según lo establecido en la Normativa de Comités Curriculares de la Vicerrectoría Académica.
		\item Asistir a los Consejos Académicos.
		\item Estudiar y desarrollar proyectos y propuestas académicas para el alumnado. También organizar actividades dirigidas al bienestar de los estudiantes representados, y mantener constante contacto con estos, para prevenir y estar al tanto de ``cosas de proceso de permanencia/causal''.% TODO Rewrite
		\item Representar a los estudiantes representados en el resto de los espacios pertinentes al cargo según estime la contingencia para la correcta gestión de este.
	\end{enumerate}
\end{art}

\begin{art}\label{funcionesConsejeroTerritorial}
	Son funciones y atribuciones del Consejero Territorial que represente a la Facultad de Matemática todos aquellos dispuestos en los estatutos FEUC, y aquellos dispuestos en este estatuto.
\end{art}

\begin{art}\label{funcionesConsejoEstudiantil}
	Son funciones  y atribuciones del Consejo Estudiantil:
	\begin{enumerate}
		\item Elegir un director de Consejo y un secretario.
		\item Velar por el buen trabajo del CAM.
		\item Proponer y votar sobre proyectos que el CAM proponga.
		\item En caso de que el Consejo lo estime necesario, destituir a cualquier miembro de un organismo directivo del CAM. Esto se regirá según lo descrito en el artículo \ref{destituciones} en la sección \nameref{vacancias}
	\end{enumerate}
\end{art}

\section{De las Asambleas y Consejos}\label{asambleasConsejos}
\begin{art}\label{asambleas}
	Las Asambleas tienen carácter informativo y/o consultivo, pudiendo transformarse en resolutivas si es aprobado por la mayoría simple de los presentes en la Asamblea. En el caso de que estas sean resolutivas, deberán contar con una asistencia mínima del 35\% de los estudiantes representados. Hay dos tipos de Asambleas:
	\begin{enumerate}
		\item Asamblea Ordinaria: Realizar al menos dos cada semestre, la cual deberá ser citada con al menos dos días hábiles de anticipación.
		\item Asamblea Extraordinaria: Convocada por la Directiva o cualquier grupo de mínimo 7 estudiantes representados, previa entrega de sus firmas al CAM. Esta deberá ser citada con al menos tres días hábiles de anticipación.% TODO Rewrite first sentence
	\end{enumerate}
\end{art}

\begin{art}\label{consejos}
	Los Consejos Estudiantiles serán citados por el director del Consejo, estos deben convocarse como mínimo una vez cada mes, exceptuando los meses de Enero, Febrero y Julio, y con al menos dos días hábiles de anticipación.
\end{art}

\section{De la elección de cargos}\label{elecciones}
\begin{art}\label{eleccionesConvocación}
	Será responsabilidad de la directiva en ejercicio convocar a las elecciones correspondientes.
\end{art}

\subsection{De la elección de la Directiva}\label{eleccionesCAM}
\begin{art}\label{}
	La Directiva CAM se elegirá anualmente por votación directa, libre, secreta e informada.
\end{art}

\begin{art}\label{eleccionesCAMListaUnica}
	En caso de que solo haya una postulación, la votación deberá ser de apruebo o rechazo. % TODO rewrite last sentence
\end{art}

\begin{art}\label{eleccionesCAMPostulacion}
	Cada postulación tendrá que designar integrante por cada posición de la Directiva, además para formalizar la inscripción esta se debe hacer llegar a la directiva del CAM en ejercicio la lista con todos los integrantes.
\end{art}

\begin{art}\label{eleccionesCAMFechas}
	Las elecciones se harán a lo menos cinco días hábiles después del término del proceso de inscripción. Este último se deberá realizar durante la segunda semana de Octubre y debe durar 5 días hábiles, la fecha exacta será determinada por la directiva en ejercicio, con a lo menos 5 días hábiles de antelación.
\end{art}

\begin{art}\label{eleccionesCAMPublicacion}
	Al término del proceso de inscripción el Secretario General deberá publicar en los medios oficiales las listas de candidatos y sus respectivos programas.
\end{art}

\begin{art}\label{eleccionesCAMInscripcion}
	Si una vez cumplido el plazo de inscripción, no hay inscripción, el consejo estudiantil decidirá si agregará un plazo extraordinario a lo más de cinco días hábiles o designar una directiva interina.
\end{art}

\begin{art}\label{eleccionesCAMFalla}
	En caso de no tener inscripción o de rechazo de una única lista, la directiva interina compuesta por miembros del consejo estudiantil, tomará la directiva del CAM hasta una nueva convocatoria el siguiente año, junto con la elección de delegados.
\end{art}

\begin{art}\label{eleccionesCAMCandidatos}
	Sólo podrán ser candidatos a los cargos de la directiva aquellos estudiantes de pregrado que sean integrantes de la Facultad de Matemáticas y que no estén en proceso de permanencia. % TODO añadir cosas del proceso de permanencia
\end{art}

\begin{art}\label{eleccionesCAMVotacion}
	Para que la elección tenga validez se debe cumplir con los puntos descritos en la sección \ref{votaciones}
\end{art}

\subsection{De la elección de la Consejería Académica}\label{eleccionesConsejeria}

\begin{art}\label{eleccionesConsejeriaLista}
	Cada postulación a la Consejería Académica será conformada por un Consejero y un Subconsejero académico.
\end{art}

\begin{art}\label{eleccionesConsejeriaFalla}
	Si al final del plazo de inscripción no hay postulaciones, se agregará un plazo extraordinario a lo más de cinco días hábiles. En caso de no haber una postulación en este plazo extraordinario, el vicepresidente interno tomará el rol de Consejero Académico.
\end{art}

\begin{art}\label{eleccionesConsejeriaVotacion}
	La elección se realizará en conjunto con la de la directiva CAM, por lo que cual deberán seguir la condiciones establecidas en la sección \ref{votaciones}.
\end{art}

\subsection{De la elección de los Delegados}\label{eleccionesDelegados}

\begin{art}\label{eleccionesDelegadosFecha}
	La elección de Delegados será la segunda semana de abril, la fecha la dispondrá la directiva al mando y no durará más de cinco días hábiles.
\end{art}

\begin{art}\label{eleccionesDelegadosCandidatos}
	Todo estudiante de la Facultad de Matemáticas podrá postularse a ser Delegado de su generación. La elección se conformará por votaciones por generación. Entendiéndose generación de la forma descrita en el artículo \ref{definicionesOrganismos}.% ? refrasear 
\end{art}

\begin{art}\label{eleccionesDelegadosPostulacionUnica}
	En caso de que en alguna generación solo haya una postulación, la votación correspondiente será de apruebo o rechazo.
\end{art}

\begin{art}\label{eleccionesDelegadosFalla}
	En caso de que en alguna generación no exista una postulación, el Secretario General deberá suplir las funciones del Delegado correspondiente hasta que se consiga algún representante.
\end{art}

\section{Del Tribunal Calificador de Elecciones (TRICEL)}\label{TRICEL}
% * TODO Rearmar el TRICEL para que no sea temporal
\begin{art}\label{TRICELMision}
	El TRICEL es el organismo encargado de organizar, vigilar y concretar todo proceso eleccionario a nivel del CAM y, como tal, tiene carácter temporal.
\end{art}

\begin{art}\label{TRICELDescripcion}
	Serán características del TRICEL:
	\begin{enumerate}
		\item Estar compuesto por un representante del CAM en ejercicio.
		\item Estar compuesto por un estudiante perteneciente al Consejo Estudiantil.
		\item En votación de Directiva se permitirá que una persona de cada lista forme parte del TRICEL.
		\item El CAM coordinará la conformación del TRICEL.
	\end{enumerate}
\end{art}

\begin{art}\label{TRICELFunciones}
	Serán funciones del TRICEL:
	\begin{enumerate}
		\item Velar por la realización y garantizar la transparencia del proceso eleccionario.
		\item Conocer la cualquier asunto relacionado con la elección para la cual se constituye.
		\item Calificar la elección dando su dictamen respecto de la legitimidad o nulidad parcial o total del proceso.
		\item Determinar la lista de personas con derecho a voto de acuerdo a lo estipulado en el presente estatuto.
		\item Atender, investigar y resolver los reclamos y observaciones presentadas con respecto al proceso eleccionario por cualquier estudiante.
		\item Determinar, publicar y distribuir el material necesario para la implementación de del acto eleccionario.
		\item Si fuese necesario, contemplar lo referente a la mecánica de votación a través de un reglamento para dicha elección.
	\end{enumerate}
\end{art}

\begin{art}\label{TRICELFecha}
	El TRICEL deberá constituirse a lo menos con una semana de anticipación a todo proceso eleccionario.
\end{art}

\begin{art}\label{TRICELResolucion}
	Las resoluciones del TRICEL serán tomadas por acuerdo unánime y solo serán apelables ante el mismo tribunal por vía de reconsideración.
\end{art}

\begin{art}\label{TRICELExcepciones}
	El cargo de miembro del TRICEL será incompatible con el de candidato para la elección en que se constituye dicho tribunal.
\end{art}

\section{De Vacancias, Destituciones y Renuncias}\label{vacancias}
\subsection{Vacancias}\label{vacancias}
% TODO Rellenar con lo del grupo de la Angela

\subsection{Destituciones}\label{destituciones}
% TODO Rellenar con las especificaciones bajo las cuales se pierde un cargo, o es destituido por el Consejo Estudiantil
	
\subsection{Renuncias}\label{renuncias}
% TODO Rellenar con los plazos y acciones a tomar bajo renuncias.

\section{De las Votaciones}\label{votaciones}
% TODO Rellenar con los mínimos para toda votación

\section{Del Plebiscito}\label{plebiscito}
% ? REDO

\begin{art}\label{plebiscitoDescripcion}
	El plebiscito es la consulta  directa a todos los estudiantes sobre materias especificas y que tiene carácter vinculante al interior del CAM.
\end{art}

\begin{art}\label{plebiscitoAntelacion}
	La convocatoria a plebiscito debe ser precisa con alternativas claras e informadas, presentadas con al menos cinco días hábiles de antelación al plebiscito.
\end{art}

\begin{art}\label{plebiscitoConvocar}
	Pueden convocar a Plebiscito:
	\begin{enumerate}
		\item La Directiva del CAM por unanimidad.
		\item La Asamblea General con mayoría absoluta de sus votos.
		\item El Consejo Estudiantil con mayoría absoluta de sus votos. % TODO check if vote system is compatible with other stuff
	\end{enumerate}
\end{art}

\begin{art}\label{plebiscitoValidez}
	Para que el plebiscito sea válido, deberá sufragar el 40\% de los estudiantes representados.
	% TODO check for other conditions that may be needed or changed.
	% ? Should we add something in case its not valid? A new date, or something?
\end{art}

\begin{art}\label{}
	Resultará ganadora aquella alternativa que logre obtener la mayoría simple de los votos válidamente emitidos. % ? Should we move this to the section about votes? Or should we reference it?
\end{art}

\begin{art}\label{}
	El plebiscito estará a cargo de la Directiva del CAM, y el TRICEL ofrecerá encargados de mesa, los cuales serán testigos en los recuentos parciales y totales. % TODO check with rewrite of TRICEL
\end{art}


\section{De los Estatutos}\label{estatutos}% ? Mover, agregar y/o reescribir cosas de la sección

\begin{art}\label{}
	Podrán proponer reformas a los presentes estatutos:
	\begin{enumerate}
		\item La Asamblea con mayoría absoluta de los votos.
		\item El Consejo Estudiantil.
		\item Cualquier miembro del CAM.
	\end{enumerate}
\end{art}

\begin{art}\label{}
	Toda proposición debe ser presentada por escrito a la Asamblea General acompañada de la argumentación que justifica la solicitud de la reforma. La Asamblea General decidirá por mayoría absoluta de la totalidad de sus miembros si dicha proposición es o no materia obligada de plebiscito.
\end{art}

% ? Considerar convenciones en la documentación y los cambios de los estatutos, además de la creación de una organización y un repositorio correspondiente en GitHub o algún servicio equivalente, para facilitar las futuras reformas y la documentación de los cambios.
\begin{art}\label{}
	Se deberá mantener un historial de cambios y propuestas, tanto aprobadas como rechazadas, de los estatutos en la plataforma GitHub (). La administración de este historial será responsabilidad del Secretario General, el cual deberá usar el sistema de “pull requests” y “branches” para mantener de manera ordenada los cambios realizados y propuestos, detallándolos de la mejor manera posible.
\end{art}

\begin{art}\label{}
	En caso de existir cualquier duda sobre la interpretación del presente estatuto la Asamblea General será quien deba decidir dicha interpretación.
\end{art}

% * La siguiente parte corresponde a la documentación de futuras y la presente reforma.

\vfill
\textsc{Comité sobre la reforma de los estatutos}\\ % ? Reescribir
Facultad de Matemáticas, Pontificia Universidad Católica de Chile\\
Santiago, Noviembre de 2019

\newpage

\begin{sloppypar}
	\textsc{Elaborado por}% TODO Rellenar con los nombres del comité y de las personas que hagan reformas a futuro.

\end{sloppypar}
\end{document}
